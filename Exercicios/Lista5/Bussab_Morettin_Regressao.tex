\documentclass{article}

\usepackage{Sweave}
\begin{document}
\Sconcordance{concordance:Bussab_Morettin_Regressao.tex:Bussab_Morettin_Regressao.Rnw:%
1 2 1 1 0 2 1 1 7 48 1}




\begin{center}
{\large MAF 105 - Iniciação à Estatística}\\
\end{center}

\section*{Exercícios do Bussab e Morettin.}

$\newline$

\subsection*{Capítulo 16 — Regressão Linear Simples}

\begin{itemize}
\item Página 454 - Exercícios: 1 ao 4. 
\item Página 26 - Exercícios: 9, 11, 16, 18.
\end{itemize}

\subsection*{Capítulo 3 — Medidas -Resumo}

\begin{itemize}
\item Páginas 40 e 41 - Exercícios: 1 ao 6.
\item Página 47 - Exercício: 7.
\item Página 50 - Exercício: 11.
\item Páginas 56 a 67 - Exercícios: 14, 19, 21, 22, 23, 25, 26, 29, 30, 32, 33, 38 e 40.
\end{itemize}

\subsection*{Capítulo 4 — Análise Bidimensional}

\begin{itemize}
\item Páginas 72 a 73 - Exercícios: 1 ao 3.
\item Página 89 - Exercício: 11.
\item Páginas 94 a 99 - Exercícios: 18, 19, 24, 25, 29.
\end{itemize}

% << >>=
% df2 <- data.frame(Var = c("V1", "V2","V3","V1","V2","V3"),
%                   Esp=rep(c("E1", "E2"),each=3),
%                   Resp=c(8, 10, 12, 6, 8, 10) )
% head(df2)
% library(ggplot2)
% # Line plot with multiple groups
% ggplot(data=df2, aes(x=Var, y=Resp, group=Esp)) +
%   geom_line(aes(linetype=Esp))+
%   geom_point()+
%   theme(legend.position="top")+ylab("Altura de Plantas (cm)")
% @


\end{document}
