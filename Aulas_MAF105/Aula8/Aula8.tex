\documentclass[14pt,aspectratio=1610]{beamer}

\usepackage[brazil]{babel}
\usepackage[utf8]{inputenc}
%\UseRawInputEncoding
\usepackage[T1]{fontenc}
\usepackage{Sweave}
\usepackage{animate}
\usepackage{amsbsy}
\usepackage{amsfonts}
\usepackage{amsmath}
\usepackage{amssymb}
\usepackage{amsthm}
\usepackage[toc,page,title,titletoc]{appendix}
\usepackage[fixlanguage]{babelbib}
%\usepackage[pdftex]{color}
\usepackage{dsfont}
\usepackage{esvect}
\usepackage[labelfont=bf]{caption}
\usepackage{float}
\usepackage[Glenn]{fncychap}%Sonny %Conny %Lenny %Glenn %Renje %Bjarne %Bjornstrup
%\usepackage{geometry, calc, color, setspace}%
%\geometry{a4paper, headsep=1.0cm, footskip=1cm, lmargin=3cm, rmargin=2cm, tmargin=3cm, bmargin=2cm}
\usepackage{graphicx}
\usepackage{indentfirst}%Para indentar os parágrafos automáticamente
\usepackage{lipsum}
\usepackage{longtable}
\usepackage{mathtools}
\usepackage{listings}%Inserir codigo do R no latex
\usepackage{multirow}
\usepackage{multicol}
\usepackage{natbib}
\bibliographystyle{abbrvnat3}
\usepackage[figuresright]{rotating}
\usepackage{spalign}
%\usepackage{pgfpages}
\usepackage{pgfplots}
\usepackage{tikz}
\usepackage{color, colortbl}
\usepackage{ragged2e}%para justificar o texto dentro de algum ambiente
\definecolor{Gray}{gray}{0.9}
\definecolor{LightCyan}{rgb}{0.88,1,1}


\usepackage[all]{xy}
\usepackage{hyperref,bookmark}
\hypersetup{
  colorlinks=true,
  linkcolor=blue,
  citecolor=red,
  filecolor=blue,
  urlcolor=blue,
}

\usetheme{Montpellier}
%\usecolortheme[RGB={193,0,0}]{structure}

%\setbeamertemplate{footline}[frame number]
%\setbeamertemplate{footline}[text line]{%
%  \parbox{\linewidth}{\vspace*{-8pt}\hfill\date{}\hfill\insertshortauthor\hfill\insertpagenumber}}
\beamertemplatenavigationsymbolsempty
\renewcommand{\vec}[1]{\mbox{\boldmath$#1$}}
\newtheorem{Teorema}{Teorema}
\newtheorem{Proposicao}{Proposição}
\newtheorem{Definicao}{Definição}
\newtheorem{Corolario}{Corolário}
\newtheorem{Demonstracao}{Demonstração}
\newcommand{\bx}{\ensuremath{\bar{x}}}
\newcommand{\Ho}{\ensuremath{H_{0}}}
\newcommand{\Hi}{\ensuremath{H_{1}}}


\apptocmd{\frame}{}{\justifying}{} % Allow optional arguments after frame.

\title{MAF 105 - Iniciação à Estatística}
\author{Prof. Fernando de Souza Bastos}
\institute{Instituto de Ciências Exatas e Tecnológicas\texorpdfstring{\\ Universidade Federal de Viçosa}{}\texorpdfstring{\\ Campus UFV - Florestal}{}}
\date[\today]{}
\newcommand\mytext{Aula de Variáveis Aleatórias}
\newcommand\mytextt{Fernando de Souza Bastos}
\makeatletter
\setbeamertemplate{footline}
{
  \leavevmode%
  \hbox{%
  \begin{beamercolorbox}[wd=.333333\paperwidth,ht=2.25ex,dp=1ex,center]{author in head/foot}%
    \usebeamerfont{author in head/foot}\mytext
  \end{beamercolorbox}%
  \begin{beamercolorbox}[wd=.333333\paperwidth,ht=2.25ex,dp=1ex,center]{title in head/foot}%
    \usebeamerfont{title in head/foot}\mytextt
  \end{beamercolorbox}%
  % \begin{beamercolorbox}[wd=.333333\paperwidth,ht=2.25ex,dp=1ex,right]{date in head/foot}%
  %   \usebeamerfont{date in head/foot}\insertshortdate{}\hspace*{2em}
  %   \insertframenumber{} / \inserttotalframenumber\hspace*{2ex} 
  % \end{beamercolorbox}
  }%
  \vskip0pt%
}
\makeatother


\providecommand{\arcsin}{} \renewcommand{\arcsin}{\hspace{2pt}\textrm{arcsen}}
\providecommand{\sin}{} \renewcommand{\sin}{\hspace{2pt}\textrm{sen}}
%\newtheorem{Teorema}{Teorema}
%\newtheorem{Proposicao}{Proposição}
%\newtheorem{Definicao}{Definição}
%\newtheorem{Corolario}{Corolário}
%\newtheorem{Demonstracao}{Demonstração}

% Layout da pagina
\hypersetup{pdfpagelayout=SinglePage}
\begin{document}
\Sconcordance{concordance:Aula8.tex:Aula8.Rnw:%
1 324 1}


\frame{\titlepage}

\begin{frame}{}
\frametitle{\bf Sumário}
\tableofcontents
\end{frame}

\section{Variáveis Aleatórias Bidimensionais}
\begin{frame}{}
\frametitle{}
\begin{block}{}
\justifying
No estudo de variáveis aleatórias, até este ponto, considerou-se que o resultado do experimento em questão seria registrado como um único valor x. Todavia, existem casos em que há interesse por dois resultados simultâneos, como por exemplo observar o peso e altura de uma pessoa, o sexo e peso de um recém-nascido, etc. Para tanto, faz-se necessário a seguinte definição:

Sejam $E$ um experimento aleatório, e $\Omega$ o espaço amostral associado a $E$.
Sejam $X$ e $Y$ duas variáveis aleatórias. Então $(X, Y )$ define uma variável aleatória bidimensional, que pode ser discreta, contínua ou mista.
\end{block}
\end{frame}

\section{Distribuição Conjunta de duas variáveis aleatórias}
\begin{frame}{}
\frametitle{}
\begin{block}{}
\justifying
Se $(X, Y )$ é uma variével aleatória bidimensional discreta, sua função de
probabilidade, representada por $P(X = x_{i}, Y = y_{j})$ que associa um valor $p(x_{i}, y_{j})$ a cada
valor do par $(X, Y )$ deve satisfazer as seguintes condições:

\begin{enumerate}
\item $P(x_{i}, y_{j})\geq 0, \forall (x_{i}, y_{j});$
\item $\displaystyle \sum_{i} \sum_{j} P(x_{i}, y_{j}) = 1.$
\end{enumerate}
\end{block}
\end{frame}

\begin{frame}{}
\frametitle{}
\begin{block}{Exemplo}
\justifying
\begin{enumerate}
\item Seja o experimento de se lançar simultaneamente um dado e uma moeda, observando o resultado da face superior de ambos. Teremos então a seguinte
função de probabilidade, onde:
$$X= \textrm{face superior do dado,\ e}\ \ Y=\textrm{face superior da moeda}$$
\end{enumerate}
\vspace{-0.8cm}
\begin{table}[h]
\centering
\resizebox{0.3\textwidth}{!}{
\begin{tabular}{ccc}
  \hline
  $X\backslash Y$ & cara & coroa \\
  \hline
  1 & $1/12$ & $1/12$ \\
  2 & $1/12$ & $1/12$ \\
  3 & $1/12$ & $1/12$ \\
  4 & $1/12$ & $1/12$ \\
  5 & $1/12$ & $1/12$ \\
  6 & $1/12$ & $1/12$ \\
  \hline
\end{tabular}
}
\end{table}
\end{block}
\end{frame}

\begin{frame}{}
\frametitle{}
\begin{block}{}
\justifying
Se $(X, Y )$ for uma variável aleatória bidimensional contínua, diz-se que
$f(x, y)$ é uma função densidade de probabilidade conjunta se:

\begin{enumerate}
\item $f(x, y)\geq 0, \forall\ (x_{i}; y_{j})\in \mathbb{R};$
\item $\displaystyle \int_{-\infty}^{\infty} \int_{-\infty}^{\infty} f(x, y)dxdy = 1.$
\end{enumerate}
\end{block}
\end{frame}

\begin{frame}{}
\frametitle{Exemplo}
\begin{block}{}
\vspace{-1cm}
\begin{enumerate}
\item Sejam $X$ e $Y$ v.a.c. com f.d.p. conjunta dada por:
\end{enumerate}
$$
f(x,y)=\left\{
\begin{array}{ccc}
k(2x+y),& \textrm{se}\quad 2\leq x\leq 6,\quad 0\leq y\leq 5;\\
0,& \textrm{para outros valores de\ }X\ \textrm{e}\ Y.\\
\end{array}
\right.
$$
Pede-se:
\begin{description}
\item[a)] O valor de $k$;
\item[b)] $P(X\leq 3, 2\leq Y \leq 4)$;
\item[c)] $P(Y<2)$;
\item[d)] $P(X>4)$;
\end{description}
\end{block}
\end{frame}

\section{Distribuição Marginal}
\begin{frame}{Distribuição Marginal}
\frametitle{}
\begin{block}{}
\justifying
Dada uma variável aleatória bidimensional, e sua distribuição de probabilidade conjunta, pode-se obter a distribuição da variável $X$, sem considerar $Y$ ou vice-versa, 
que são denominadas distribuições marginais de $X$ e $Y$, respectivamente.

\begin{itemize}
\item Distribuição marginal de $X$, caso em que $X$ é v.a.d.:
$$\displaystyle P(X=x_{i})=\sum_{j}p(x_{i},y_{j})$$
\end{itemize}
\end{block}
\end{frame}

\begin{frame}{}
\frametitle{}
\begin{block}{}
\begin{itemize}
\item Distribuição marginal de $X$, caso em que $X$ é v.a.c.:
$$\displaystyle g(x)=\int_{-\infty}^{\infty} f(x,y)dy$$\pause
\vspace{-0.5cm}
\item Distribuição marginal de $Y$, caso em que $Y$ é v.a.d.:
$$\displaystyle P(Y=y_{j})=\sum_{i}p(x_{i},y_{j})$$\pause
\vspace{-0.5cm}
\item Distribuição marginal de $Y$, caso em que $Y$ é v.a.c.:
$$\displaystyle h(y)=\int_{-\infty}^{\infty} f(x,y)dx$$
\end{itemize}
\end{block}
\end{frame}

\begin{frame}{}
\frametitle{Exemplo}
\begin{block}{}
No exemplo do lançamento simultâneo de um dado e uma moeda teremos:
$$X= \textrm{face superior do dado,\ e}\ \ Y=\textrm{face superior da moeda}$$
\begin{table}[h]
\centering
\resizebox{0.4\textwidth}{!}{
\begin{tabular}{c|cc|c}
  \hline
  $X\backslash Y$ & cara & coroa&$P(X=x_{i})$ \\
  \hline
  1 & $1/12$ & $1/12$& $1/6$\\
  2 & $1/12$ & $1/12$& $1/6$\\
  3 & $1/12$ & $1/12$& $1/6$\\
  4 & $1/12$ & $1/12$& $1/6$\\
  5 & $1/12$ & $1/12$& $1/6$\\
  6 & $1/12$ & $1/12$& $1/6$\\
  \hline
$P(Y=y_{j})$ &$1/2$& $1/2$&1\\
  \hline
\end{tabular}
}
\end{table}
\end{block}
\end{frame}

\section{Variáveis Aleatórias Independentes}
\begin{frame}{}
\frametitle{Variáveis Aleatórias Independentes}
\begin{block}{}
\justifying
Seja $(X, Y)$ uma variável aleatória bidimendional, então as variáveis $X$ e $Y$
são independentes se, e somente se, 
$$P(x_{i}, y_{j}) = P(x_{i}).P (y_{j}),\quad \forall i,j,$$

para variáveis aleatórias discretas, ou 

$$f(x, y) = g(x).h(y),\quad \forall i,j,$$

para variáveis aleatórias contínuas. 

\end{block}
\end{frame}

\begin{frame}{}
\frametitle{}
\begin{block}{}
\justifying
O exemplo anterior é um caso de v.a. independentes, basta notar que 
$P(x_{i}, y_{j}) =\dfrac{1}{12}= \dfrac{1}{6}\dfrac{1}{2}=P(x_{i}).P (y_{j}),\quad \forall i,j,$
\end{block}
\end{frame}

\begin{frame}{}
\frametitle{}
\begin{block}{}
\justifying
As v.a. $X$ e $Y$ admitem a seguinte distribuição conjunta de probabilidade.
\begin{center}
\begin{tabular}{c|ccc|c}
    $Y\backslash  X$ & 1 & 2 & 3 & P(y) \\
  \hline
  4    & 0,2 & 0,15 & b    &   \\
  5    & a   & 0,15 & 0,15 &   \\
  \hline
  P(x) &     &      &      &   \\
   \end{tabular}
\end{center}

Encontre $a$ e $b$ para que as v.a. $X$ e $Y$ sejam independentes.
\end{block}
\end{frame}


\end{document}
